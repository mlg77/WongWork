\section{Option A} \label{Sect.A}

\noindent From Figure \ref{fig.rbc.approx}a Option A assumes the cell wall is negligible. Given the cell wall thickness, $t_{wall} << r_{RBC}$ it can be assumed that the cell wall plays a negligible role in the surface stability. This is the least physiological of the options as realistically this is not the case. However, there are two main upsides to this assumption. First, rupture criteria does not need to be considered as the dynamics would be the same post rupture and second, the mathematical analysis is directly comparable to the water droplet example in Section \ref{Sect.SE}. 


Here, the surface stability equation (amplitude of perturbation equation) will be re-examined in the context of a \ac{rbc} and associated parameters. Unlike the water droplet in air where by the two mediums have extremely different properties, the blood plasma and the cytoplasm have very comparable properties. Cytoplasm differs from blood plasma primarily by the presence of hemoglobin (Section \ref{Sect.Blood}). Table \ref{table.plasmas} shows the similarities between the cytoplasm and the blood plasma properties.
\\
\begin{table} [H]
	\begin{center}
		\caption{ Summary of main similarities between cytoplasm and the blood plasma properties. \ }
		\label{table.plasmas}
		\begin{tabularx} {0.85\textwidth}{L{4cm} L{1.5cm} | C{3.5cm} C{3.5cm}  }
			\hline
			\textbf{Property (Symbol)} & \textbf{Units} & \textbf{Cytoplasm} & \textbf{Blood Plasma} \\
			\hline
			Viscosity ($\upmu$) & mPa$\cdot$s & $2.27-9.87$ \cite{Tomaiuolo2014} & $1.1–1.3$ \cite{Kesmarky2008} \\
			Density ($\rho$) & kgm$^{-3}$ & $1125 $ \cite{bloodfact2004} & $1025$  \cite{bloodfact2004}  \\
			
			
			\hline			
		\end{tabularx}
	\end{center}
\end{table}

\noindent The two fluids are very similar compared to the differences between water and air and thus, the same simplifications made in Section \ref{Sect.SE} can not be made. Importantly, it is possible that the cytoplasm and the blood plasma will mix due to the comparable densities. This is a major complication and one that needs to be thoroughly considered when choosing this option. For the purpose of this research and to fully explore this option, it is assumed that the two fluids are immiscible (will not mix) on the specified time scale. 


Equation \ref{eq.der.big2} can be simplified to suit a \ac{hs} \ac{rbc} given the similarities in properties in Table \ref{table.plasmas}. First,  $\rho_1 \approx \rho_2$ therefore, $\rho_1 \approx \rho_2 \equiv \rho = 1075$ kgm$^{-3}$. Second, the viscosities will be taken as the average of the ranges, thus, the densities of the cytoplasm and the blood plasma will be $\upmu_1 = 6.07$ mPa$\cdot$s and $\upmu_2 = 1.2$ mPa$\cdot$s respectively. Finally, the small differences between $\upmu_1$ and $\upmu_2$ will lead to even smaller differences between $T_1(R,t)$ and $T_2(R,t)$ and thus, can be approximated as equal, ie $T_1(R,t) \approx T_2(R,t) \equiv T(R,t)$. 


\begin{equation} \label{eq.opta.der1}
\begin{split}
&\left[ \frac{1}{n} + \frac{1}{n+1} \right] \rho \ddot{a} + \left[3 \rho \left( \frac{1}{n} + \frac{1}{n+1} \right) \frac{\dot{R}}{R} - 2(n-1)(n+2) \frac{\upmu_2 - \upmu_1}{R^2} \right]\dot{a}  \\ & +\left[ \rho \left(  \frac{n+2}{n} - \frac{n-1}{n+1} \right)\frac{\ddot{R}}{R} + (n-1)(n+2)\frac{\gamma}{R^3} + 2(n-1)(n+2)(\upmu_2 -\upmu_1)\frac{\dot{R}}{R^3}  \right] a  \\& + T(R,t) \left[ (n-1)(n+1)\frac{\upmu_1}{R} - n(n+2)\frac{\upmu_2}{R} \right]  \\ & - (n+1)\rho_1\dot{R}R^{-n-3} \int_0^R (s^3-R^3)s^{n-1}T(s,t)ds \\ & + n\rho_2 \dot{R}R^{n-2} \int_{R}^{\infty} (s^3 - R^3)s^{-n-2} T(s,t) ds = 0 
\end{split}
\end{equation}

\noindent Next the same boundary conditions are employed such that Equation \ref{eq.der.T} and \ref{eq.der.intT} still hold true. Thus Equation \ref{eq.opta.der1} will now become Equation \ref{eq.opta.der2}.

\begin{equation} \label{eq.opta.der2}
\begin{split}
&\left[ \frac{1}{n} + \frac{1}{n+1} \right] \rho \ddot{a} + \left[3 \rho \left( \frac{1}{n} + \frac{1}{n+1} \right) \frac{\dot{R}}{R} - 2(n-1)(n+2) \frac{\upmu_2 - \upmu_1}{R^2} \right]\dot{a}  \\ & +\left[ \rho \left(  \frac{n+2}{n} - \frac{n-1}{n+1} \right)\frac{\ddot{R}}{R} + (n-1)(n+2)\frac{\gamma}{R^3} + 2(n-1)(n+2)(\upmu_2 -\upmu_1)\frac{\dot{R}}{R^3}  \right] a  \\& + \frac{2}{n}\left[(n+1)\dot{a} - (n+2)\frac{\dot{R}}{R} a\right] \left[ (n-1)(n+1)\frac{\upmu_1}{R} - n(n+2)\frac{\upmu_2}{R} \right]  = 0 
\end{split}
\end{equation}

\noindent which is rearranged to

\begin{equation} \label{eq.opta.der3}
\begin{split}
&\left[ \frac{(2n+1)\rho}{n(n+1)}\right]  \ddot{a} + \left[3 \frac{(2n+1)\rho}{n(n+1)}  \frac{\dot{R}}{R} - 2(n-1)(n+2) \frac{\upmu_2 - \upmu_1}{R^2} \right]\dot{a}   \\& + \frac{2}{nR}(n+1)\left[ (n-1)(n+1)\upmu_1 - n(n+2)\upmu_2 \right]\dot{a} \\ & +\left[ \rho  \frac{4n+2}{n(n+1)} \frac{\ddot{R}}{R} + (n-1)(n+2)\frac{\gamma}{R^3} + 2(n-1)(n+2)(\upmu_2 -\upmu_1)\frac{\dot{R}}{R^3}  \right] a \\& - \frac{2}{n} (n+2)\frac{\dot{R}}{R^2} \left[ (n-1)(n+1)\upmu_1 - n(n+2)\upmu_2 \right]a   = 0 
\end{split}
\end{equation}

\noindent Finally, similar to Equation \ref{eq.der.adotdot}, Equation \ref{eq.opta.der3} can be written as  Equation \ref{eq.opta.adotdot}.

\begin{equation}\label{eq.opta.adotdot}
\ddot{a} + B_n(t) \dot{a} - A_n(t) a = 0 
\end{equation}
\noindent where
\begin{equation}\label{eq.opta.An}
\begin{split}
A_n(t) = & 2 \frac{\ddot{R}}{R} + \frac{n(n+1)}{\rho (2n+1)} \left[ \frac{(n-1)(n+2)}{R^3}\left(\gamma + 2(\upmu_2 -\upmu_1)\dot{R} \right)  - \right. \\ & \left. \frac{2(n+2)\dot{R}}{R^2} \left( \left(n+\frac{1}{n}\right)\upmu_1 - (n+2)\upmu_2 \right) \right]
\end{split}
\end{equation}
\noindent and
\begin{equation}\label{eq.opta.Bn}
\begin{split}
B_n(t) = & 3 \frac{\dot{R}}{R} + \frac{n(n+1)}{\rho (2n+1)} \left[ 2(1-n)(n+2) \frac{\upmu_2 - \upmu_1}{R^2} + \right. \\ & \left. \frac{2(n+1)}{R}\left(\left(n-\frac{1}{n}\right)\upmu_1 - (n+2)\upmu_2 \right) \right]
\end{split}
\end{equation}


\noindent Thus, Equation \ref{eq.der.adotdot} can now be written as Equation \ref{eq.opta.adotdot} to suite a \ac{rbc} with \ac{hs}. ie. the final amplitude of perturbation of the surface is $a = f(t, n, \rho, \mu_1, \mu_2, \gamma, R, \dot{R}, \ddot{R})$.


