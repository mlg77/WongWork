\section{Concluding Remarks and Future Work}
Here \ac{rbc}'s with \acf{hs} (sphering of the \ac{rbc} causing a reduced diameter) was investigated as a premise to apply surface stability (or amplitude of perturbation) equations to. The primary objective of this research was to formulate a surface stability equation for a \ac{rbc} similar to that described by \citet{Prosperetti1974} and further used by \citet{Zeng2018}. However, it became quickly apparent that the surface stability equation, based upon surface tension, was not directly comparable to a \ac{rbc} and thus, approximations were needed.

This research presented three different approaches to approximate the physiology of a \ac{rbc}; A) negligible cell wall, B) gel cell wall and C) thin shell cell wall mechanical deformation. Options A and B were considered less physiological than C, but, they were able to apply a similar surface stability equation as the source material. Option B and C, although more physiological than option A, were required to also consider cell rupture. Option C was the most physiological of the options, however, this required an alternative approach looking at a the \acf{dmv} thin shell wall approximation to the deformation. The resulting equation included a spatial consideration not previously included and, thus, would significantly increase computation time.   

The three options provide a good range of physiology and computations scale. Unfortunately however, without raw data of a \ac{rbc} under deformation there is no way of validating the provided equations. Additionally, the amplitude of perturbation equations provided here would need to be incorporated within the entire sphere dynamics described in \citet{Zeng2018} to fully understand their impact. This extensive additional research is expected to have multiple difficulties, some of which were described in this research.   

Overall, the results included within this research are a step towards the combination of these two research areas. It is difficult to say which option presented here would be the most suitable going forward as the benefits of physiological accuracy need to be weighted against the disadvantages of computation time and approximations. It is hoped that one day \acf{hs} \ac{rbc}'s can be accurately modelled in order to advance medications to alleviate some of the symptoms associated with \ac{hs}.
